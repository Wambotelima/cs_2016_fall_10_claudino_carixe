\documentclass{article}
\usepackage[utf8]{inputenc}

\title{English literature history}
\author{claudiocarixe }
\date{DECEMBER 2016}

\usepackage{natbib}
\usepackage{graphicx}

\begin{document}

\maketitle

\section{Introduction}

The story of English literature begins with the Germanic tradition of the Anglo-Saxon settlers. Beowulf stands at its head. 

This epic poem of the 8th century is in Anglo-Saxon, now more usually described as Old English. It is incomprehensible to a reader familiar only with modern English. Even so, there is a continuous linguistic development between the two. The most significant turning point, from about 1100, is the development of Middle English - differing from Old English in the addition of a French vocabulary after the Norman conquest. French and Germanic influences subsequently compete for the mainstream role in English literature. 
 	









The French poetic tradition inclines to lines of a regular metrical length, usually linked by rhyme into couplets or stanzas. German poetry depends more on rhythm and stress, with repeated consonants (alliteration) to bind the phrases. Elegant or subtle rhymes have a courtly flavour. The hammer blows of alliteration are a type of verbal athleticism more likely to draw applause in a hall full of warriors. 

Both traditions achieve a magnificent flowering in England in the late 14th century, towards the end of the Middle English period. Piers Plowman and Sir Gawain are masterpieces which look back to Old English. By contrast Chaucer, a poet of the court, ushers in a new era of English literature.

\section{development}
Of these two great English alliterative poems, the second is entirely anonymous and the first virtually so. The narrator of Piers Plowman calls himself Will; occasional references in the text suggest that his name may be Langland. Nothing else, apart from this poem, is known of him. 

Piers Plowman exists in three versions, the longest amounting to more than 7000 lines. It is considered probable that all three are by the same author. If so he spends some twenty years, from about 1367, adjusting and refining his epic creation. 
 	









Piers the ploughman is one of a group of characters searching for Christian truth in the complex setting of a dream. Though mainly a spiritual quest, the work also has a political element. It contains sharply observed details of a corrupt and materialistic age (Wycliffe is among Langland's English contemporaries).

Where Piers Plowman is tough and gritty, Sir Gawain and the Green Knight (dating from the same period) is more polished in its manner and more courtly in its content. The characters derive partly from Arthurian legend. 
 	







A mysterious green knight arrives one Christmas at the court of King Arthur. He invites any knight to strike him with an axe and to receive the blow back a year later. Gawain accepts the challenge. He cuts off the head of the green knight, who rides away with it. 

The rest of the poem concerns Gawain, a year later, at the green knight's castle. In a tale of love (for the green knight's wife) and subsequent deceit, Gawain emerges with little honour. The green knight spares his life but sends him home to Arthur's court wearing the wife's girdle as a badge of shame.



\section{Conclusion}

The Scrutiny movement's history divides into three periods: roughly, 1930–40, 1940–65, and 1965 onwards. The movement was at its fullest extent in the middle period, and benefited from the distinctive conjuncture of the post-war period, a moment of cultural democratization and the persistence of older hierarchies and standards. As this double helix of democratization and deference unwound, the Scrutiny principle of ‘discrimination’ lost its external sanction. The decline of the movement after the early 1960s is also attributable to the way its cultural critique had been pushed more or less to exhaustion by the time the Birmingham Centre was founded. The conclusion takes stock of the divergent political positions informed by the Scrutiny tradition, and suggests that the movement matters for that diversity as well as for its impact on education and its role in bringing popular culture within the purview of the humanities in Britain.
{English literature}

\bibliographystyle{plain}
\bibliography{references}
\end{document}
